\documentclass[letterpaper]{article}
\input{helper/header} %Formatting, macros, package list, etc

\title{Syllabus}
%\subtitle{}
%\assignedDate{2016-08-30}
%\dueDate{2016-09-06}
\studentName{}
%\linenumbers

\usepackage{longtable}
\usepackage{import}
\begin{document}

\begin{center}
\vspace{-18\lineskip}
\begin{tabular}{ll||ll}
\textbf{Instructor}:        & Dr. Scott Cook                                        & \textbf{Sec (010)}:     & TR 11:50-1:05\\
\textbf{My Email}:          & \href{mailto:scook@tarleton.edu}{scook@tarleton.edu}  & \textbf{Office Hours}:  & MTWRF 10:30-11:30 or by appointment\\
\textbf{My Office}:         & Math Bldg 132 (254)968-1958                           & \textbf{Math Dept Office}:  & Math Bldg 142 (254)968-9168\\
\end{tabular}
\hhrule
\end{center}

\bu[Course Content]: This is a continuation of Data Mining 1 Math 5364 where we learned many of the essential ideas of data mining.  We worked with small, clean, pre-packaged data sets so that we could focus on the ideas.  Data mining 2 is where the rubber hits the road.  We will build upon the first semester foundation in many ways.  Some goals:
\bit
	\item Handle messy data
	\item Handle big data
	\item Scrape data
	\item Master existing powerful tools to make this work computationally feasible.  Some examples:
	\bit
		\item Python and its scientific stack
		\item Data wrangling tools 
		\item Web scrapers
		\item Cloud computing
		\item Parallel processing
		\item Text mining/NLP techniques		
		\item Tensor Flow		
		\item Hadoop
		\item Visualization tools
	\eit
\eit
This list is subject to change according to the interests of the instructor and students.

\bu[Materials]: The internet is rich with open, high quality materials.  Whereas last semester we followed one primary textbook, this semester we will cast our nets wide and try to pull in lots of different sources.

As a starting point, we will use 2 excellent, free books written by Jake Vanderplas.
\bit
	\item Python Data Science Handbook:  \href{https://github.com/jakevdp/PythonDataScienceHandbook}{\underline{https://github.com/jakevdp/PythonDataScienceHandbook}}.  Consider supporting free resources by buying at \href{http://shop.oreilly.com/product/0636920034919.do}{\underline{http://shop.oreilly.com/product/0636920034919.do}}
	\item Whirlwind Tour of Python:  \href{https://github.com/jakevdp/WhirlwindTourOfPython}{\underline{https://github.com/jakevdp/WhirlwindTourOfPython}}.  Consider downloading for free at \href{http://www.oreilly.com/programming/free/a-whirlwind-tour-of-python.csp}{\underline{http://www.oreilly.com/programming/free/a-whirlwind-tour-of-python.csp}}  (I suspect he gets some credit when we download).
\eit

\bu[Seminar Style]: We will adopt a seminar style for this course where the students have substantial sway over the topics we cover and lead much of the in-class discussion.  Early in the semester, we will make a list of topics and divvy them up to groups of 2-3.  That group will research the topic, collect a library of materials, write a short book chapter about the topic, lead lecture, and design homework on the topic.  I hope to compile these chapters into a book for future courses.

\bu[Homework]: We will again use GitHub to manage assignments.
 
\bu[Grading Policy]: The guaranteed grade cutoffs are listed below.  At my sole discretion, I may curve the course by relaxing them at the end of the semester.

\begin{center}
	\begin{tabular}{|c|c|} \hline
		Homework & 30\%\\ \hline
		Topic Presentations & 40\%\\ \hline
		Final Project & 30\%\\\hline
	\end{tabular}
	\quad
	\begin{tabular}{|c|c|} \hline
		A & $[90\%,100\%]$\\ \hline
		B & $[80\%,90\%)$\\ \hline
		C & $[70\%,80\%)$\\ \hline
		D & $[60\%,70\%)$\\ \hline
		F & $[0\%,60\%)$\\ \hline
	\end{tabular}
\end{center}

\hhrule
\begin{center}
\textbf{\LARGE{University Master Syllabus}}\\
\href{http://catalog.tarleton.edu/syllabus}{\underline{http://catalog.tarleton.edu/syllabus}}
\end{center}

\small
\bu[Catalog Description]: This course centers on the identification, exploration, and extraction of new patterns from natural language text documents using appropriate software. Selected topics will be chosen from association analysis, anomaly detection, text mining, dimensionality reduction, and model evaluation and comparison.

\bu[Student Learning Objectives]: Upon completing this course, a student should be able to do the following: 
\ben[a)]
\item Examine raw data in order to detect data quality issues and interesting subsets or features contained within the data.
\item Transform raw data into a form appropriate for modeling.
\item Select and train appropriate models using the transformed data.
\item Measure the effectiveness of each model.
\item Draw appropriate conclusions.
\een

\bu[University Policy]: Students are responsible for knowing and abiding by the policies and information contained in the Tarleton Student Handbook [TSUSH].

\bu[Student Responsibilities]:  The student is solely responsible for:
\bit
	\item	Attending class.
	\item Completing every assignment by the specified due date.
	\item Utilizing, as needed, all available study-aid options (including meeting with the instructor, attending Supplemental Instruction (SI) sessions, going to the Math Clinic, using tutorial software, purchasing a student solutions manual, hiring a personal tutor, etc.) to resolve any questions that they might have regarding homework, course material, and/or technology projects.
	\item	Reading all relevant material in the course text and lecture.
	\item	Being present and prepared for each exam on the specified date and time, unless the instructor determines that a makeup exam is warranted (see Makeup Policy above).
  \item Obtaining assignments and other materials for classes from which they are absent.
  \item Giving as much effort as it takes to pass this course.
\eit

\bu[Services for Students with Disabilities]:  It is the policy of Tarleton State University to comply with the Americans with Disabilities Act and other applicable laws. If you are a student with a disability seeking accommodations for this course, please contact the Center for Access and Academic Testing at 254.968.9400 or caat@tarleton.edu. The office is located in Math 201. More information can be found at www.tarleton.edu/caat, in the University Catalog, or at [www.ada.gov]www.ada.gov.

\bu[Cell phones]: Students are expected to set their cell phone so as to emit no audible noise in the classroom. Except for emergency situations, cell phone use (including texting) during the class period is prohibited. A student who is noticeably (to the instructor) distracted by his/her cell phone and/or distracting others with it may be asked to immediately disable it or to leave the classroom.  To compensate for your electronic deprivation, keep your calculator on.

\bu[Absence Policy]:  Class absence policies will be established and enforced by each individual course instructor.  The course instructor may recommend to the Dean of Students that a student be dropped from a course if excessive absences prevent satisfactory progress.

\bu[Makeup Policy]:  Each course instructor has the responsibility and authority to determine if work can be made-up because of absences.  Students may request make-up considerations for valid and verifiable reasons such as the following:
\bit
\item Illness
\item Death in the immediate family
\item Legal proceedings
\item Participation in sponsored University activities (It is the responsibility of students who participate in University-sponsored activities to obtain a written explanation for their absence from the faculty/staff member who is responsible for the activity.)
\eit

\bu[Drop Policy]:  A student who withdraws from a course before the thirteenth class day of a regular semester or before the fifth class day in a summer term receives no grade, and the course will not be listed on that student's permanent record.  A student who withdraws from a course before the end of the tenth week of a regular semester or the fourteenth class day of a summer term receives a grade of W.

Tarleton differentiates between a failed grade in a class because a student never attended (F0 grade), stopped attending at some point in the semester (FX grade), or because the student did not pass the course (F) but attended the entire semester. These grades will be noted on the official transcript. Stopping or never attending class is considered an unofficial withdrawal and can result in the student having to return aid monies received.  For more information see the Tarleton Financial Aid website.


\bu[Day of Service - April 7th] - In support of Tarleton's core value of service, each student is expected to participate in a service learning experience as a part of the Spring term week of service.  This experience will challenge students to be engaged in the local community, address a community need, connect course objectives to the world around you, and involve structured student reflection. In this service learning experience you will not only enhance your knowledge and skills, but actively use those skills as you serve your community.


\bu[Student Safety and Title IX]: You are in college to achieve academic success, but you must feel safe and take care of yourself to reach your full potential. You have the right to pursue your education in a safe environment. Title IX makes it clear that violence and harassment based on sex and gender are civil rights offenses subject to accountability. \textit{If you or someone you know has been harassed or assaulted, there is help and support on campus}. You may seek assistance confidentially through the Student Counseling Center or the Student Health Center. You may also make a report to the campus Title IX coordinator, which may trigger a university investigation (not a criminal investigation). Additionally, you may pursue criminal charges through the university police department. If the assault occurred away from campus, UPD can assist you in connecting with the appropriate law enforcement agency.
\begin{center}
Student Counseling Center: 254-968-9044 (phone is answered 24 hours a day, 7 days a week), TSC 212

Student Health Services: 254-968-9271, TSC 212 

Title IX Coordinator: 254-968-9754, Admin Annex 1, Room 112

University Police Department: 254-968-9002, located on the back side of Wisdom Gym
\end{center}

\bu[Academic Integrity Statement]:  Tarleton State University's core values are integrity, leadership, tradition, civility, excellence, and service.  Central to these values is integrity, which is maintaining a high standard of personal and scholarly conduct.  Academic integrity represents the choice to uphold ethical responsibility for one's learning within the academic community, regardless of audience or situation.

\bu[Academic Civility Statement]:  Students are expected to interact with professors and peers in a respectful manner that enhances the learning environment. Professors may require a student who deviates from this expectation to leave the face-to-face (or virtual) classroom learning environment for that particular class session (and potentially subsequent class sessions) for a specific amount of time. In addition, the professor might consider the university disciplinary process (for Academic Affairs/Student Life) for egregious or continued disruptive behavior.

\bu[Academic Excellence Statement]:  Tarleton holds high expectations for students to assume responsibility for their own individual learning.  Students are also expected to achieve academic excellence by:
\bit
  \item honoring Tarleton's core values. 
  \item upholding high standards of habit and behavior.
  \item maintaining excellence through class attendance and punctuality.
  \item preparing for active participation in all learning experiences. 
  \item putting forth their best individual effort.
  \item continually improving as independent learners.
  \item engaging in extracurricular opportunities that encourage personal and academic growth.
  \item reflecting critically upon feedback and applying these lessons to meet future challenges.
\eit

\bu[Academic Conduct]:  Tarleton State University expects its students to maintain high standards of personal and scholarly conduct.  Students guilty of academic dishonesty are subject to disciplinary action.  Academic dishonesty includes, but is not limited to, cheating on an examination or other academic work, plagiarism, collusion, and the abuse of resource materials.  The faculty member is responsible for initiating action for each case of academic dishonesty that occurs in his/her class.  Students guilty of academic dishonesty, cheating, or plagiarism in academic work shall be subject to disciplinary action.  The instructor may initiate disciplinary action in any case of academic misconduct.

\bu[Academic dishonesty] includes, but is not limited to, cheating on an examination or other academic work, plagiarism, collusion, unauthorized use of technology and the abuse of resource materials.
\ben
  \item Academic work means the preparation of an essay, thesis, problem, assignment or other projects submitted or completed for course credit and to meet other requirements for noncourse credit.
  \item What constitutes an act of academic dishonesty may, in part, depend on the particular course and expectations of academic integrity in the context of the course objectives. This includes, but is not limited to, the following:
  \ben
    \item Copying, without instructor authorization, from another student's test paper, laboratory report, other report, computer fi les, data listing and/or programs.
    \item Using, during a test, materials not authorized by the person giving the test. 
    \item Collaborating with another person without instructor authorization during an examination or in preparing academic work.
    \item Knowingly and without instructor authorization, using, buying, selling, stealing, transporting, soliciting, copying, or possessing, in whole or in part, the contents of an unadministered test or other required assignment.
    \item Substituting for another student or permitting another person to substitute for oneself in taking an examination, preparing academic work, or attending class.
    \item Bribing another person to obtain an unadministered test or information about an unadministered test.
    \item Using technological equipment such as calculators, computers or other electronic aids in taking of tests or preparing academic work in ways not authorized by the instructor or the university.
  \een
  \item Plagiarism means the appropriation of another's work and the unacknowledged incorporation of that work in one's own written work in any academic setting. 
  \item Collusion means the unauthorized collaboration with another person in preparing written work in any academic setting.
  \item Abuse of resource materials means the mutilation, destruction, concealment, theft or alteration of materials provided.
\een

\begin{center}
	\textbf{This syllabus subject to change as deemed appropriate by the instructor.}
\end{center}

\end{document}